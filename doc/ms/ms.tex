% Options for packages loaded elsewhere
\PassOptionsToPackage{unicode}{hyperref}
\PassOptionsToPackage{hyphens}{url}
%
\documentclass[
]{article}
\usepackage{lmodern}
\usepackage{setspace}
\usepackage{amsmath}
\usepackage{ifxetex,ifluatex}
\ifnum 0\ifxetex 1\fi\ifluatex 1\fi=0 % if pdftex
  \usepackage[T1]{fontenc}
  \usepackage[utf8]{inputenc}
  \usepackage{textcomp} % provide euro and other symbols
  \usepackage{amssymb}
\else % if luatex or xetex
  \usepackage{unicode-math}
  \defaultfontfeatures{Scale=MatchLowercase}
  \defaultfontfeatures[\rmfamily]{Ligatures=TeX,Scale=1}
\fi
% Use upquote if available, for straight quotes in verbatim environments
\IfFileExists{upquote.sty}{\usepackage{upquote}}{}
\IfFileExists{microtype.sty}{% use microtype if available
  \usepackage[]{microtype}
  \UseMicrotypeSet[protrusion]{basicmath} % disable protrusion for tt fonts
}{}
\usepackage{xcolor}
\IfFileExists{xurl.sty}{\usepackage{xurl}}{} % add URL line breaks if available
\IfFileExists{bookmark.sty}{\usepackage{bookmark}}{\usepackage{hyperref}}
\hypersetup{
  pdftitle={Manuscript title here},
  hidelinks,
  pdfcreator={LaTeX via pandoc}}
\urlstyle{same} % disable monospaced font for URLs
\usepackage[margin=1in]{geometry}
\usepackage{graphicx}
\makeatletter
\def\maxwidth{\ifdim\Gin@nat@width>\linewidth\linewidth\else\Gin@nat@width\fi}
\def\maxheight{\ifdim\Gin@nat@height>\textheight\textheight\else\Gin@nat@height\fi}
\makeatother
% Scale images if necessary, so that they will not overflow the page
% margins by default, and it is still possible to overwrite the defaults
% using explicit options in \includegraphics[width, height, ...]{}
\setkeys{Gin}{width=\maxwidth,height=\maxheight,keepaspectratio}
% Set default figure placement to htbp
\makeatletter
\def\fps@figure{htbp}
\makeatother
\setlength{\emergencystretch}{3em} % prevent overfull lines
\providecommand{\tightlist}{%
  \setlength{\itemsep}{0pt}\setlength{\parskip}{0pt}}
\setcounter{secnumdepth}{-\maxdimen} % remove section numbering
\usepackage{lineno}
\usepackage{amsmath}
\usepackage{indentfirst}
\linenumbers
\ifluatex
  \usepackage{selnolig}  % disable illegal ligatures
\fi
\newlength{\cslhangindent}
\setlength{\cslhangindent}{1.5em}
\newlength{\csllabelwidth}
\setlength{\csllabelwidth}{3em}
\newenvironment{CSLReferences}[2] % #1 hanging-ident, #2 entry spacing
 {% don't indent paragraphs
  \setlength{\parindent}{0pt}
  % turn on hanging indent if param 1 is 1
  \ifodd #1 \everypar{\setlength{\hangindent}{\cslhangindent}}\ignorespaces\fi
  % set entry spacing
  \ifnum #2 > 0
  \setlength{\parskip}{#2\baselineskip}
  \fi
 }%
 {}
\usepackage{calc}
\newcommand{\CSLBlock}[1]{#1\hfill\break}
\newcommand{\CSLLeftMargin}[1]{\parbox[t]{\csllabelwidth}{#1}}
\newcommand{\CSLRightInline}[1]{\parbox[t]{\linewidth - \csllabelwidth}{#1}\break}
\newcommand{\CSLIndent}[1]{\hspace{\cslhangindent}#1}

\title{Manuscript title here}
\author{true}
\date{}

\begin{document}
\maketitle

\setstretch{1}
\newpage

\hypertarget{abstract}{%
\section{Abstract}\label{abstract}}

\newpage

\hypertarget{introduction}{%
\section{Introduction}\label{introduction}}

\setlength\parindent{24pt}

Temperature is an important abiotic variable with increasing relevance
as warming global temperatures alter the diversity, structure, and
functioning of Earth's ecosystems (Walther et al. 2002). Temperature's
effect on ecosystems manifest through complex direct and indirect
pathways such as shifting species ranges {[}{]}{[}{]} and subsequent
changes to local and regional communities {[}{]}, species adaptations
(Gibert and DeLong 2017), and through effects on individual metabolic
rates (Gillooly et al. 2001, Brown et al. 2004). While a growing body of
theoretical and empirical study has enhanced our knowledge of
temperature-mediated changes to ecosystems (O'Connor et al. 2009),
general patterns are uncertain and empirical studies often idiosyncratic
(Nelson et al. 2017a, Zhang et al. 2017), especially at higher levels of
organizations such as communities and food webs (Walther et al. 2002,
Woodward et al. 2010). Yet, identifying general effects of temperature
is a necessary step towards understanding how future warming may alter
the functioning of ecosystems and the services they provide.

\par

\setlength\parindent{24pt}

Temperature has the potential to alter the provision and maintenance of
ecosystem services by modifying the interactions among species (Woodward
et al. 2010, Brose et al. 2012) that underpin ecosystem functions (de
Ruiter et al. 1995, Thompson et al. 2012). The signature of temperature
on the stability and dynamics of food webs is present across global
climate gradients (Baiser et al. 2019) where its direct and indirect
effects alter the magnitudes and relative distributions among species
and trophic levels (May 1972, McCann et al. 1998, Barnes et al. 2018).
Direct changes through species richness effects (Gibert 2019). Indirect
through the distribution of consumer and prey species and it effects on
the acquisition and allocation of resources among species (Zhang et al.
2017) --macroecology of food web dynamics and stability are controlled
by the .

\par

\setlength\parindent{24pt}

Temperature can directly alter the distribution of species interactions
by asymmetric responses in organismal traits among species {[}e.g.,
attack rate, handling time, growth rates, etc.; Dell et al. (2014){]}.

Metabolic rates

acquisition of resources by consumer Generally, temperature is predicted
to have disproportionately higher effects on consumers relative to
resources (Allen et al. 2005, Vasseur and McCann 2005,
\textbf{oconnor2011?})

Here, we measured the patterning and distribution of organic matter
fluxes within invertebrate food webs across a natural stream temperature
gradient (\textasciitilde5 - 28\(^\circ\)C). Previous research in these
streams has shown a strong positive effect of temperature on primary
production both among streams (Padfield et al. 2017,
\textbf{demars2011?}) and within streams seasonally (O'Gorman et al.
2012, Hood et al. 2018). Consumers rely largely on autochthonous
resource (O'Gorman et al. 2012, \textbf{nelson2019?}) and therefore the
dynamics of primary production have a strong control on consumer energy
demand (Junker et al. 2020), as such, we predicted total annual OM
fluxes to consumers to scale with among stream patterns in primary
production and consumer energy demand, and therefore, increase with
temperature across streams. We, further expected that increasing
temperature would reduce consumer species (\textbf{ogorman2019?}),
thereby altering how OM fluxes are distributed within and across
communities. The distribution of OM fluxes to consumers will shift
towards relatively faster, higher turnover consumer at higher
temperatures, `speeding up' consumer dynamics both through direct
effects on consumer turnover and through a decrease in mean body size.
Seasonally, warmer streams will lead to earlier food web fluxes in
warmer compared to colder streams.

\hypertarget{methods}{%
\section{Methods}\label{methods}}

We studied six streams within the Hengill geothermal field of
southwestern Iceland (64\(^\circ\) 03'N 021\(^\circ\) 18'W) that varied
in mean annual temperature. Hengill is characterized by indirect
geothermal heating of groundwater (Arnason et al. 1969), leading to a
natural variability in water temperatures (4.5--54.0 \(^\circ\)C), but
similar solute chemistries (Friberg et al. 2009). These conditions
create a ``natural laboratory'' for isolating the effects of temperature
on ecosystem processes (O'Gorman et al. 2014, Nelson et al. 2017b). We
selected streams to maximize the temperature range, while minimizing
differences in the structural aspects of primary producers. In each
stream, we measured temperature and water depth every 15 min from July
2010 through August 2012 (U20-001-01 water-level logger, Onset Computer
Corp.~Pocasset, MA, USA). Light availability in the watershed was
measured every 15 min from atmospheric stations (HOBO pendant
temperature/light UA-002-64, Onset Computer Corp.~Pocasset, MA, USA).

\hypertarget{invertebrate-sampling}{%
\subsection{Invertebrate sampling}\label{invertebrate-sampling}}

We sampled macroinvertebrate communities approximately monthly from July
2011 to August 2012 in four streams and from October 2010 to October
2011 in two streams used in a previous study (\emph{n} = 6 streams). The
two streams were part of an experiment beginning October 2011, therefore
overlapping years were not used to exclude the impact of experimental
manipulation (see Nelson et al. 2017a, 2017b). Inter-annual comparisons
of primary and secondary production in previous studies showed minimal
differences among years in unmanipulated streams, suggesting that
combining data from different years would not significantly bias our
results (Nelson et al. 2017a, Hood et al. 2018). We collected fiver
Surber samples (0.023 m\textsuperscript{2}, 250 \(\mu\)m mesh) from
randomly selected locations within each stream. Within the sampler,
inorganic substrates were disturbed to \textasciitilde10 cm depth and
invertebrates and organic matter were removed from stones with a brush.
Samples were then preserved with 5\% formaldehyde until laboratory
analysis. In the laboratory, we split samples into coarse (\textgreater1
mm) and fine (\textless1 mm but \textgreater250 \(\mu\)m) fractions
using nested sieves and then removed invertebrates form each fraction
under a dissecting microscope (10--15 x magnification). For particularly
large samples, fine fractions were subsampled (1/2--1/16th) using a
modified Folsom plankton splitter prior to removal of invertebrates.
Subsamples were scaled to the rest of the sample assuming similar
abundance and body size distributions. Macroinvertebrates were
identified to the lowest practical taxonomic level (usually genus) with
taxonomic keys (Peterson 1977, Merritt et al. 2008, Andersen et al.
2013). Taxon-specific abundance and biomass were scaled to a per meter
basis by dividing by the Surber sampler area.

\hypertarget{secondary-production}{%
\subsection{Secondary Production}\label{secondary-production}}

Daily secondary production of invertebrate taxa was calculated using the
instantaneous growth rate method {[}IGR; Benke and Huryn (2017){]}.
Growth rates were determined using taxon appropriate approaches
described in (Junker et al. 2020). Briefly, growth rates of common taxa
(e.g., Chironomidae spp., Radix balthica, etc.) were determined using
\emph{in situ} chambers (Huryn and Wallace 1986). Multiple individuals
(\emph{n} = 5--15) within small size categories (\textasciitilde1 mm
length range) were photographed next to a field micrometer, placed into
the stream within pre-conditioned chambers for 7--15 days, after which
they were again photographed. Individual lengths were measured from
field pictures using image analysis software (Schindelin et al. 2012),
and body lengths were converted to mass (mg ash-free dry mass
{[}AFDM{]}) using published length-mass regressions (Benke et al. 1999,
O'Gorman et al. 2012, Hannesdóttir et al. 2013). Growth rates (\emph{g},
d\textsuperscript{-1}) were calculated by the changes in mean body size
(\emph{W}) over a given time interval (\emph{t}) with the following
equation:

\[g = log_e ( W_{t+\Delta t} / W_t) / \Delta t\]

Variability in growth rates was estimated by bootstrapping through
repeated resampling of individual lengths with replacement (\emph{n} =
500). For taxa which exhibit synchronous growth and development (e.g.,
Simuliidae spp., some Chironomidae spp., etc.), we examined temporal
changes in length-frequency distributions and calculated growth rates
and uncertainty using a bootstrap technique similar to that described in
Benke and Huryn (2017). Individual Lengths were converted to mg AFDM
using published length-mass regression cited above and size-frequency
histograms were visually inspected for directional changes in body size
through time. For each date, size-frequency distributions were resampled
with replacement (\emph{n} = 500) and growth rates estimated from
equation 1. We prevented the calculation of negative growth rates by
requiring \(W_{t + \Delta t}\) \textgreater{} \(W_t\). To estimate
growth rates of taxa for which growth could not be estimated
empirically, we developed stream-specific growth rate models by
constructing multivariate linear regressions of empirical growth data
against body size and temperature within each stream. To estimate
uncertainty in production of each taxon, we used a bootstrapping
technique that resampled measured growth rates, in addition to abundance
and size distributions from individual samples. For each iteration
(\emph{n} = 1000), size-specific growth rates were multiplied by mean
interval biomass for each size class and the number of days between
sample dates to estimate size class-specific production. For each
interval, size classes were summed for each taxon to calculate total
population-level production. Intervals were summed to estimate annual
secondary production.

\hypertarget{diet-analysis}{%
\subsection{Diet analysis}\label{diet-analysis}}

\textbf{Come back and confirm this section} Macroinvertebrate diets were
quantified for dominant taxa in each stream. We focused on numerically
abundant taxa and/or taxa with relatively high annual production. A
minimum of five individuals were selected from samples, and, when
possible included individuals of different size classes to account for
ontogenetic shifts in diet. We included individuals from different
seasons to capture concurrent ontogenetic and seasonal changes. For
small-bodied taxa, we combined multiple individuals (\emph{n} = 3--5) to
ensure samples contained enough material for quantification. We used
methods outlined in Rosi-Marshall (2016) to remove gut tracts and
prepare gut content slides. Briefly, we removed the foregut from each
individual or collection of individuals and sonicated gut contents in
water for 30 seconds. Gut content slurries were filtered onto gridded
nitrocellulose membrane filters (Metricel GN-6, 25 mm, 0.45 \(\mu\)m
pore size; Gelman Sciences, Ann Arbor, MI, USA), dried at 60 \(^\circ\)C
for 15 min, placed on a microscope slide, cleared with Type B immersion
old, and covered with a cover slip. We took 5--10 random photographs
under 200--400x magnification, depending on the density of particles,
using a digital camera mounted on a compound microscope. From these
photographs we identified all particles within each field and measured
the relative area of particles using image analysis software (Schindelin
et al. 2012). We identified particles as diatoms, green and filamentous
algae, cyanobacteria, amorphous detritus, vascular and non-vascular
plants (bryophytes), and animal material. We calculated the proportion
of each food category in the gut by dividing their summed area by the
total area of all particles. Gut contents of many predators were empty
or contained unidentifiable, macerated prey. For these taxa, we assumed
100\% animal material.

To estimate variability in diet compositions and to impute missing
values for non-dominant, yet present, taxa, we modeled the diet
proportions within each stream using a

hierarchical multivariate model (\textbf{coblentz2017?}). Here, the diet
of a consumer population, \emph{i}, is a multinomial
vector,\(\overrightarrow{y_i}\), of
\[\overrightarrow{y_i} \sim Multinomial(\overrightarrow{p_i}, n_i)\]
\[\overrightarrow{p_i} \sim Dirichlet(\overrightarrow{q_i} \times  \alpha) \]
where, \(\overrightarrow{p_i}\), is a vector of consumer diet
proportions, \(\overrightarrow{q_i}\) is a vector of the population's
diet proportions and \(\alpha\) is a concentration parameter of the
Dirichlet process. We used uniform priors for \(\overrightarrow{q_i}\)
and \(\alpha\),
\[\overrightarrow{q_i} \sim Dirichlet(\overrightarrow{\textbf{1}})\]
\[\alpha \sim Uniform(0,\textit{c})\] where,
\(\overrightarrow{\textbf{1}}\) is a vector of ones the same length of
basal resource types and \(\textit{c}\) is the presumed maximum value
the concentration value. \textbf{add in site-level model parameter and
model fitting language} . Models were fit in Stan with the `brms'
package in R (Bürkner 2017). For non-dominant taxa, diet proportions
were imputed from the hierarchical model by resampling from posterior
distributions. Importantly, this process allowed to maintain the
hierarchical structure of the data when imputing missing values.

From modeled diet compositions, we estimated trophic redundancy within
and across stream food webs by calculating proportional similarities
{[}\emph{PS}; (\textbf{whittaker1952?}){]} among modeled diet estimates
(proportional similarity estimates from empirical diets give similar
results to modeled estimates; Supporting Materials). Proportional
similarities were calculated as:

\[ PS = 1 - 0.5 \sum_{j=1}^S|p_{x,i} - p_{y,i}|\]

where, \emph{p\textsubscript{x,i}} is the proportion of food resource
\emph{i} in the diet of taxon \emph{x}, \emph{p\textsubscript{y,i}} is
the proportion of food resource \emph{i} in the diet of taxon \emph{y},
and there are \emph{S} food categories. Proportional similarity was
calculated across all taxa within a stream based on modeled diet
contributions from each taxon. To calculate \emph{PS} among streams we
sampled 1000 estimates of the mean stream-level diet proportions for
each stream and calculated \emph{PS} for each.

\hypertarget{organic-matter-consumption-estimates}{%
\subsection{Organic Matter Consumption
Estimates}\label{organic-matter-consumption-estimates}}

Consumption fluxes (g m\textsuperscript{-2} t\textsuperscript{-1}) were
calculated using the trophic basis of production method (TBP; (Benke and
Wallace 1980). Taxon-specific secondary production estimates were
combined with

diet proportions, diet-specific assimilation efficiencies,
\emph{AE\textsubscript{i}}, and assumed net production efficiencies,
\emph{NPE}, to estimate consumption of organic matter. For each food
category, \emph{i}, diet proportions were multiplied by the gross growth
efficiency (\(GGE_{i} = AE_{i} * NPE\)) to calculate the relative
production attributable to each food category. The relative production
from each food type was then multiplied by the interval-level production
and finally divided by \(GGE_{i}\) to estimate consumption of organic
matter from each food category by a consumer (Benke and Wallace 1980).
Consumption was calculated for each taxon across sampling intervals
(typically \textasciitilde1 month). Total interval consumption was
calculated by summing across all taxa, while annual consumption was
calculated by summing across all taxa and intervals. Variability in
consumption estimates was estimated through a Monte Carlo approach,
wherein bootstrapped vectors of secondary production for each taxon (see
\emph{Secondary production} methods above) were resampled and
consumption estimated with the TBP method using modeled diet proportions
(see \emph{Diet analysis} above), diet-specific assimilation
efficiencies, and net production efficiency. Variability in
\emph{AE\textsubscript{i}} was incorporated by resampling values from
beta distributions fit to median and 2.5\% and 97.5\% percentiles for
each diet item: diatoms = 0.30 (95\% percentile interval (PI):
0.24-0.36), filamentous and green algae = 0.30 (95\% PI: 0.24-0.36),
cyanobacteria = 0.10 (95\% PI: 0.08-0.12), amorphous detritus = 0.10
(95\% PI: 0.08-0.12), vascular and non-vascular plants (bryophytes) =
0.1 (95\% PI:: 0.08-0.12), and animal material = 0.7 (95\% PI:
0.56-0.84)(Welch 1968, Benke and Wallace 1980, 1997, Cross et al. 2007,
2011). Variability in \emph{NPE} was incorporated by resampling values
from an assumed beta distribution with median \emph{NPE} = 0.45 (95\% PI
= 0.4-0.5). Beta distributions were fit using the `get.beta.par()'
function within the \emph{rriskDistributions} package (Belgorodski et
al. 2017).

\hypertarget{quantifying-the-distribution-of-food-web-fluxes}{%
\subsection{Quantifying the distribution of food web
fluxes}\label{quantifying-the-distribution-of-food-web-fluxes}}

\hypertarget{evenness-among-consumers}{%
\subsubsection{Evenness Among
consumers}\label{evenness-among-consumers}}

To visualize and quantify how evenly OM fluxes were distributed among
consumers within a stream, we constructed Lorenz curves (Lorenz 1905) on
ordered relative consumption fluxes, such that in a community with \(S\)
species and the relative consumption of species \emph{i}, \(p_i\), is
ordered \(p_1 \leq p_2 \leq ... p_S\). The Lorenz curve plots how a
value, in this

case OM flux, accumulates with increasing cumulative proportion of
species. In a community with perfectly equal distribution of OM
consumption among species, the Lorenz curve is simply a straight
diagonal line. Deviation from perfect equality was calculated as the
Gini coefficient (Gini 1921), normalized for differences in \(S\) among
streams, \(G^*\) (Solomon 1975, Chao and Ricotta 2019):
\[ G^* = (2 \sum_{i = 1}^S ip_i -2)/(S-1)\]

\hypertarget{distribution-along-species-trait-axes}{%
\subsubsection{Distribution along species' trait
axes}\label{distribution-along-species-trait-axes}}

We were interested in how OM fluxes were distributed (randomly
vs.~non-randomly) in relation to species traits (i.e., body size,
\(P:B\) ratio, population biomass) across the temperature gradient. We
used a permutation analyses to test for non-random patterns in OM flux
with species traits. First, we ordered species based on within-stream
ranking of annual population traits (i.e, \(M\), \(P:B\)). We then
calculated a measure of skewness, \(Sk_{flux}\), based on quartiles of
the distribution of OM fluxes in relation to species traits as:

\[Sk_{flux} = f_{Q0.75} - 2f_{Q0.5} + f_{Q0.25}/ f_{Q0.75}-f_{Q0.25}\]

where, \(f_{Qx}\), is the cumulative flux at some quantile, \(Qx\), of
the community trait distribution. To determine the probability of
observing the skewness we observe in the data compared to random
ordering, we constructed a distribution of skewness estimates from
random species ordering while maintaining the relative distribution of
OM fluxes. The number of unique orderings of species increases to
computationally intractable numbers very quickly (e.g., \(S!\)),
therefore we permuted 1,000,000 random species orderings and calculating
the skew in the cumulative distribution of annual OM fluxes. We then use
this random distribution to estimate the probability of observing a
skewness as extreme or greater.

\hypertarget{statistical-analyses}{%
\subsection{Statistical Analyses}\label{statistical-analyses}}

Relationships between mean annual temperature (\(^\circ\)C) and mean
community \emph{M} and \emph{P:B} were assessed with bootstrapped linear
regressions. Here, \ensuremath{10^{5}} values of mean \emph{M} or
\emph{P:B} were resampled with replacement from each stream. For each
resampling event, a linear model was fit between
\emph{log\textsubscript{e}}-transformed \emph{M} or \emph{P:B} and mean
annual temperature. Response variables \emph{M} and \emph{P:B} were
transformed to meet assumptions of normality of residuals variation.

\hypertarget{results}{%
\section{Results}\label{results}}

\hypertarget{community-om-fluxes}{%
\subsection{Community OM fluxes}\label{community-om-fluxes}}

Annual community energy demand mirrored patterns of secondary production
previously reported (Junker et al. 2020). Community energy demand varied
\textasciitilde44.2-fold across streams (1.005{[}0.556 , 1.555{]} to
44.4 {[}32.4 , 57.8{]}; g AFDM \(m^{-2} y^{-1}\) mean {[}95\% percentile
interval (PI){]}) and was positively related to temperature. Total OM
flux through the consumer community was strongly associated with
community energy demand and varied from 3.9 {[}2.1,6.2{]} to
177.4{[}125.6,236.6{]} g AFDM \(m^{-2} y^{-1}\).

Patterns of OM flux among streams were most closely tied to total
consumer energy demands as consumer diets exhibited high similarity
among streams (Figure SX). Diets composition among all streams were
dominated by diatoms (44.1\% {[}c(\texttt{2.5\%} = 0),c(\texttt{97.5\%}
= 75.5){]}), amorphous detritus (17.4\% {[}c(\texttt{2.5\%} =
0),c(\texttt{97.5\%} = 32.3){]}), and green algae (13.2\%
{[}c(\texttt{2.5\%} = 0),c(\texttt{97.5\%} = 42.6){]}). Within streams,
diet similarity ranged from 0.7 (0.66 -- 0.73) to 0.75 (0.71 -- 0.79).
Among streams, diet overlap was similarly high and mean overlap among
all streams was 0.89\% (0.84 -- 0.92 95\% PI). Diet similarity of
pairwise comparisons among streams showed little differences in diet
among streams and no clear relationship with temperature (Supporting
Materials).

\hypertarget{evenness-of-om-fluxes-within-streams}{%
\subsection{Evenness of OM fluxes within
streams}\label{evenness-of-om-fluxes-within-streams}}

Generally, the material fluxes within a stream were distributed
unequally among consumers, however, this varied among streams (Figure 1;
Figure S1). The calculated inequality, i.e.~Gini coefficient, ranged
from 0.07 {[}0.05 , 0.1{]} to 0.27 {[}0.24 , 0.31{]}; Table 1).
Differences in inequality were partly attributed to consumer species
richness among streams which ranged from 14 to 34 consumers. Patterns of
material fluxes were still unequally distributed among consumers even
after correcting for differences in species richness (Normalized Gini
coefficient: 0.04 {[}0.02 , 0.07{]} to 0.25 {[}0.21 , 0.28{]}; Table 1).

Unequal distributions of fluxes within streams corresponded to very
different patterns in dominance of OM fluxes across streams (Figure 1a).
For example, in an absolute sense, \textasciitilde85\% of total flux was
contributed by 2 to 11 species (0.85\% to 0.85\% of flux, respectively).
Relatively, this flux was attributed by 3\% to 29\% of the species
assemblage.

\hypertarget{om-fluxes-along-species-trait-distributions-among-sites-and-taxa}{%
\subsection{OM fluxes along species trait distributions among sites and
taxa}\label{om-fluxes-along-species-trait-distributions-among-sites-and-taxa}}

The fluxes of OM were distributed differently across body sizes and
turnover rates (\emph{P:B}) among and within streams. Across streams,
mean body size of taxa ranged from 0.13 to 2.66 and mean annual P:B
ranged from 4.23 to 31.67 (Figure 2). Both, \emph{M} and \emph{P:B}
showed associations with mean annual stream temperature that were
negative and positive, respectively. Generally, mean \emph{M} of the
community decreased -9.5\% (95\% PI, -11.1\% -- -8\%) for each increase
in 1\(^\circ\)C. In contrast, mean population \emph{P:B} of the
community increased 18.8\% (95\% IP, 17\% -- 20.7\%) for each
1\(^\circ\)C in mean annual stream temperature. Fluxes were skewed
differentially towards larger body sizes (positive skew) or towards
smaller body sizes (negative skew) among streams, with skew estimates
with body size (\emph{M}) ranging from 0.13 to -0.84 (Figure 3).
Similarly, skew in fluxes towards high turnover taxa varied among
streams ranging from 0.25 to 1 (Figure 3).

When compared to a random ordering while also preserving the relative
distribution of fluxes among taxa, the probability of observing a
similar or more extreme skew of within stream OM fluxes in relation to
species body size ranged from 0.16 to 0.38. Similarly, the probability
of a more extremely skewed distribution in relation to population
biomass turnover ranged from 0.09 to 0.34.

\hypertarget{discussion}{%
\section{Discussion}\label{discussion}}

\hypertarget{acknowledgements}{%
\section{Acknowledgements}\label{acknowledgements}}

Jeff Wesner and Abe Kanz for code sharing and discussions on modeling
diet proportions.

\hypertarget{references}{%
\section*{References}\label{references}}
\addcontentsline{toc}{section}{References}

\hypertarget{refs}{}
\begin{CSLReferences}{1}{0}
\leavevmode\hypertarget{ref-allen2005}{}%
Allen, A. P., J. F. Gillooly, and J. H. Brown. 2005. Linking the global
carbon cycle to individual metabolism. Functional Ecology 19:202--213.

\leavevmode\hypertarget{ref-andersen2013}{}%
Andersen, T., P. S. Cranston, and J. H. Epler. 2013. Chironomidae of the
{Holarctic} region: {Keys} and diagnoses, {Part} 1. {Media Tryck},
{Lund, Sweden}.

\leavevmode\hypertarget{ref-arnason1969}{}%
Arnason, B., P. Theodorsson, S. Björnsson, and K. Saemundsson. 1969.
Hengill, a high temperature thermal area in {Iceland}. Bulletin
Volcanologique 33:245--259.

\leavevmode\hypertarget{ref-baiser2019}{}%
Baiser, B., D. Gravel, A. R. Cirtwill, J. A. Dunne, A. K. Fahimipour, L.
J. Gilarranz, J. A. Grochow, D. Li, N. D. Martinez, A. McGrew, T.
Poisot, T. N. Romanuk, D. B. Stouffer, L. B. Trotta, F. S. Valdovinos,
R. J. Williams, S. A. Wood, and J. D. Yeakel. 2019. Ecogeographical
rules and the macroecology of food webs. Global Ecology and Biogeography
28:1204--1218.

\leavevmode\hypertarget{ref-barnes2018}{}%
Barnes, A. D., M. Jochum, J. S. Lefcheck, N. Eisenhauer, C. Scherber, M.
I. O'Connor, P. de Ruiter, and U. Brose. 2018. Energy {Flux}: {The Link}
between {Multitrophic Biodiversity} and {Ecosystem Functioning}. Trends
in Ecology \& Evolution 33:186--197.

\leavevmode\hypertarget{ref-belgorodski2017}{}%
Belgorodski, N., M. Greiner, K. Tolksdorf, and K. Schueller. 2017.
{rriskDistributions}: {Fitting Distributions} to {Given Data} or {Known
Quantiles}.

\leavevmode\hypertarget{ref-benke2017}{}%
Benke, A. C., and A. D. Huryn. 2017. Secondary production and
quantitative food webs. Pages 235--254 Methods in {Stream Ecology}.
{Elsevier}.

\leavevmode\hypertarget{ref-benke1999}{}%
Benke, A. C., A. D. Huryn, L. A. Smock, and J. B. Wallace. 1999.
Length-{Mass Relationships} for {Freshwater Macroinvertebrates} in
{North America} with {Particular Reference} to the {Southeastern United
States}. Journal of the North American Benthological Society
18:308--343.

\leavevmode\hypertarget{ref-benke1980}{}%
Benke, A. C., and J. B. Wallace. 1980. Trophic basis of production among
net-spinning caddisflies in a southern appalachain stream. Ecology
61:108--118.

\leavevmode\hypertarget{ref-benke1997}{}%
Benke, A. C., and J. B. Wallace. 1997. Trophic {Basis} of {Production
Among Riverine Caddisflies}: {Implications} for {Food Web Analysis}.
Ecology 78:1132--1145.

\leavevmode\hypertarget{ref-brose2012}{}%
Brose, U., J. A. Dunne, Montoya José M., O. L. Petchey, F. D. Schneider,
and U. Jacob. 2012. Climate change in size-structured ecosystems.
Philosophical Transactions of the Royal Society B: Biological Sciences
367:2903--2912.

\leavevmode\hypertarget{ref-brown2004}{}%
Brown, J. H., J. F. Gillooly, A. P. Allen, V. M. Savage, and G. B. West.
2004. Toward a {Metabolic Theory} of {Ecology}. Ecology 85:1771--1789.

\leavevmode\hypertarget{ref-burkner2017}{}%
Bürkner, P.-C. 2017. Brms: {An R Package} for {Bayesian Multilevel
Models Using Stan}. Journal of Statistical Software 80:1--28.

\leavevmode\hypertarget{ref-chao2019}{}%
Chao, A., and C. Ricotta. 2019. Quantifying evenness and linking it to
diversity, beta diversity, and similarity. Ecology 100:e02852.

\leavevmode\hypertarget{ref-cross2011}{}%
Cross, W. F., C. V. Baxter, K. C. Donner, E. J. Rosi-Marshall, T. A.
Kennedy, R. O. Hall, H. A. W. Kelly, and R. S. Rogers. 2011. Ecosystem
ecology meets adaptive management: Food web response to a controlled
flood on the {Colorado River}, {Glen Canyon}. Ecological Applications
21:2016--2033.

\leavevmode\hypertarget{ref-cross2007}{}%
Cross, W. F., J. B. Wallace, and A. D. Rosemond. 2007. Nutrient
{Enrichment Reduces Constraints} on {Material Flows} in a
{Detritus}-{Based Food Web}. Ecology 88:2563--2575.

\leavevmode\hypertarget{ref-deruiter1995}{}%
de Ruiter, P. C., A.-M. Neutel, and J. C. Moore. 1995. Energetics,
{Patterns} of {Interaction Strengths}, and {Stability} in {Real
Ecosystems}. Science 269:1257--1260.

\leavevmode\hypertarget{ref-dell2014}{}%
Dell, A. I., S. Pawar, and V. M. Savage. 2014. Temperature dependence of
trophic interactions are driven by asymmetry of species responses and
foraging strategy. Journal of Animal Ecology 83:70--84.

\leavevmode\hypertarget{ref-fordyce2011}{}%
Fordyce, J. A., Z. Gompert, M. L. Forister, and C. C. Nice. 2011. A
{Hierarchical Bayesian Approach} to {Ecological Count Data}: {A Flexible
Tool} for {Ecologists}. PLOS ONE 6:e26785.

\leavevmode\hypertarget{ref-friberg2009}{}%
Friberg, N., J. B. Dybkjær, J. S. Olafsson, G. M. Gislason, S. E.
Larsen, and T. L. Lauridsen. 2009. Relationships between structure and
function in streams contrasting in temperature. Freshwater Biology
54:2051--2068.

\leavevmode\hypertarget{ref-gibert2019}{}%
Gibert, J. P. 2019. Temperature directly and indirectly influences food
web structure. Scientific Reports 9:5312.

\leavevmode\hypertarget{ref-gibert2017}{}%
Gibert, J. P., and J. P. DeLong. 2017. Phenotypic variation explains
food web structural patterns. Proceedings of the National Academy of
Sciences 114:11187--11192.

\leavevmode\hypertarget{ref-gillooly2001}{}%
Gillooly, J. F., J. H. Brown, G. B. West, V. M. Savage, and E. L.
Charnov. 2001. Effects of size and temperature on metabolic rate.
Science (New York, N.Y.) 293:2248--2251.

\leavevmode\hypertarget{ref-gini1921}{}%
Gini, C. 1921. Measurement of {Inequality} of {Incomes}. The Economic
Journal 31:124--126.

\leavevmode\hypertarget{ref-hannesdottir2013}{}%
Hannesdóttir, E. R., G. M. Gíslason, J. S. Ólafsson, Ó. P. Ólafsson, and
E. J. O'Gorman. 2013. Increased {Stream Productivity} with {Warming
Supports Higher Trophic Levels}. Advances in Ecological Research
48:285--342.

\leavevmode\hypertarget{ref-hood2018}{}%
Hood, J. M., J. P. Benstead, W. F. Cross, A. D. Huryn, P. W. Johnson, G.
M. Gíslason, J. R. Junker, D. Nelson, J. S. Ólafsson, and C. Tran. 2018.
Increased resource use efficiency amplifies positive response of aquatic
primary production to experimental warming. Global Change Biology
24:1069--1084.

\leavevmode\hypertarget{ref-huryn1986}{}%
Huryn, A. D., and J. B. Wallace. 1986. A method for obtaining in situ
growth rates of larval {Chironomidae} ({Diptera}) and its application to
studies of secondary Production1. Limnology and Oceanography
31:216--221.

\leavevmode\hypertarget{ref-junker2020}{}%
Junker, J. R., W. F. Cross, J. P. Benstead, A. D. Huryn, J. M. Hood, D.
Nelson, G. M. Gíslason, and J. S. Ólafsson. 2020. Resource supply
governs the apparent temperature dependence of animal production in
stream ecosystems. Ecology Letters 23:1809--1819.

\leavevmode\hypertarget{ref-lorenz1905}{}%
Lorenz, M. O. 1905. Methods of {Measuring} the {Concentration} of
{Wealth}. Publications of the American Statistical Association 9:209.

\leavevmode\hypertarget{ref-may1972}{}%
May, R. M. 1972. Will a {Large Complex System} be {Stable}? Nature
238:413--414.

\leavevmode\hypertarget{ref-mccann1998}{}%
McCann, K., A. Hastings, and G. R. Huxel. 1998. Weak trophic
interactions and the balance of nature. Nature 395:794--798.

\leavevmode\hypertarget{ref-merritt2008}{}%
Merritt, R. W., K. W. Cummins, and M. B. Berg, editors. 2008. An
{Introduction} to the {Aquatic Insects} of {North America}. Fourth.
{Kendall/Hunt Publishing Co.}, {Dubuque, IA}.

\leavevmode\hypertarget{ref-nelson2017}{}%
Nelson, D., J. P. Benstead, A. D. Huryn, W. F. Cross, J. M. Hood, P. W.
Johnson, J. R. Junker, G. M. Gíslason, and J. S. Ólafsson. 2017a. Shifts
in community size structure drive temperature invariance of secondary
production in a stream-warming experiment. Ecology 98:1797--1806.

\leavevmode\hypertarget{ref-nelson2017a}{}%
Nelson, D., J. P. Benstead, A. D. Huryn, W. F. Cross, J. M. Hood, P. W.
Johnson, J. R. Junker, G. M. Gíslason, and J. S. Ólafsson. 2017b.
Experimental whole-stream warming alters community size structure.
Global Change Biology 23:2618--2628.

\leavevmode\hypertarget{ref-oconnor2009}{}%
O'Connor, M. I., M. F. Piehler, D. M. Leech, A. Anton, and J. F. Bruno.
2009. Warming and {Resource Availability Shift Food Web Structure} and
{Metabolism}. PLOS Biology 7:e1000178.

\leavevmode\hypertarget{ref-ogorman2014}{}%
O'Gorman, E. J., J. P. Benstead, W. F. Cross, N. Friberg, J. M. Hood, P.
W. Johnson, B. D. Sigurdsson, and G. Woodward. 2014. Climate change and
geothermal ecosystems: Natural laboratories, sentinel systems, and
future refugia. Global Change Biology 20:3291--3299.

\leavevmode\hypertarget{ref-ogorman2012}{}%
O'Gorman, E. J., D. E. Pichler, G. Adams, J. P. Benstead, H. Cohen, N.
Craig, W. F. Cross, B. O. L. Demars, N. Friberg, G. M. Gíslason, R.
Gudmundsdóttir, A. Hawczak, J. M. Hood, L. N. Hudson, L. Johansson, M.
P. Johansson, J. R. Junker, A. Laurila, J. R. Manson, E. Mavromati, D.
Nelson, J. S. Ólafsson, D. M. Perkins, O. L. Petchey, M. Plebani, D. C.
Reuman, B. C. Rall, R. Stewart, M. S. A. Thompson, and G. Woodward.
2012. Impacts of {Warming} on the {Structure} and {Functioning} of
{Aquatic Communities}. Pages 81--176 Advances in {Ecological Research}.
{Elsevier}.

\leavevmode\hypertarget{ref-padfield2017}{}%
Padfield, D., C. Lowe, A. Buckling, R. Ffrench-Constant, S. Jennings, F.
Shelley, J. S. Ólafsson, and G. Yvon-Durocher. 2017. Metabolic
compensation constrains the temperature dependence of gross primary
production. Ecology Letters 20:1250--1260.

\leavevmode\hypertarget{ref-peterson1977}{}%
Peterson, B. V. 1977. Black flies of {Iceland} ({Diptera}-{Simuliidae}).
Canadian Entomologist 109:449--472.

\leavevmode\hypertarget{ref-rosi-marshall2016}{}%
Rosi-Marshall, E. J., H. A. Wellard Kelly, R. O. Hall, and K. A. Vallis.
2016. Methods for quantifying aquatic macroinvertebrate diets.
Freshwater Science 35:229--236.

\leavevmode\hypertarget{ref-schindelin2012}{}%
Schindelin, J., I. Arganda-Carreras, E. Frise, V. Kaynig, M. Longair, T.
Pietzsch, S. Preibisch, C. Rueden, S. Saalfeld, B. Schmid, J.-Y.
Tinevez, D. J. White, V. Hartenstein, K. Eliceiri, P. Tomancak, and A.
Cardona. 2012. Fiji: An open-source platform for biological-image
analysis. Nature Methods 9:676--682.

\leavevmode\hypertarget{ref-solomon1975}{}%
Solomon, D. L. 1975. A comparative approach to species diversity:7.

\leavevmode\hypertarget{ref-thompson2012}{}%
Thompson, R. M., U. Brose, J. A. Dunne, R. O. Hall, S. Hladyz, R. L.
Kitching, N. D. Martinez, H. Rantala, T. N. Romanuk, D. B. Stouffer, and
J. M. Tylianakis. 2012. Food webs: Reconciling the structure and
function of biodiversity. Trends in Ecology \& Evolution 27:689--697.

\leavevmode\hypertarget{ref-vasseur2005}{}%
Vasseur, D. A., and K. S. McCann. 2005. A {Mechanistic Approach} for
{Modeling Temperature}-{Dependent Consumer}-{Resource Dynamics}. The
American Naturalist 166:184--198.

\leavevmode\hypertarget{ref-walther2002}{}%
Walther, G.-R., E. Post, P. Convey, A. Menzel, C. Parmesan, T. J. C.
Beebee, J.-M. Fromentin, O. Hoegh-Guldberg, and F. Bairlein. 2002.
Ecological responses to recent climate change. Nature 416:389--395.

\leavevmode\hypertarget{ref-welch1968}{}%
Welch, H. E. 1968. Relationships between {Assimiliation Efficiencies}
and {Growth Efficiencies} for {Aquatic Consumers}. Ecology 49:755--759.

\leavevmode\hypertarget{ref-woodward2010b}{}%
Woodward, G., J. P. Benstead, O. S. Beveridge, J. Blanchard, T. Brey, L.
E. Brown, W. F. Cross, N. Friberg, T. C. Ings, U. Jacob, S. Jennings, M.
E. Ledger, A. M. Milner, J. M. Montoya, E. O'Gorman, J. M. Olesen, O. L.
Petchey, D. E. Pichler, D. C. Reuman, M. S. A. Thompson, F. J. F. Van
Veen, and G. Yvon-Durocher. 2010. Ecological {Networks} in a {Changing
Climate}. Pages 71--138 Advances in {Ecological Research}. {Elsevier}.

\leavevmode\hypertarget{ref-zhang2017}{}%
Zhang, L., D. Takahashi, M. Hartvig, and K. H. Andersen. 2017. Food-web
dynamics under climate change. Proceedings of the Royal Society B:
Biological Sciences 284:20171772.

\end{CSLReferences}

\end{document}
